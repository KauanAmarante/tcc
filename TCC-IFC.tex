% ------------------------------------------------------------------------
% Modelo de Trabalho de Conclusão de Curso em conformidade com 
% ABNT NBR 14724:2011: Informacao e documentacao - Trabalhos academicos -
% Apresentacao
% ------------------------------------------------------------------------

\documentclass[12pt, oneside, a4paper, brazil]{abntex2}
% ---
% Pacotes básicos 
% ---
\usepackage{lmodern}			     % Usa a fonte Latin Modern			
\usepackage[T1]{fontenc}		   % Selecao de codigos de fonte.
\usepackage[utf8]{inputenc}		 % Codificacao do documento (conversão automática dos acentos)
\usepackage{lastpage}			     % Usado pela Ficha catalográfica
\usepackage{indentfirst}		   % Indenta o primeiro parágrafo de cada seção.
\usepackage{color,xcolor}			 % Controle das cores
\usepackage{graphicx}			     % Inclusão de gráficos
\usepackage{microtype} 			   % para melhorias de justificação
\usepackage[alf]{abntex2cite}	 % Citações padrão ABNT
\usepackage{lipsum}            % Pode ser removido no final
\usepackage{listingsutf8}



% Altera o nome padrão do rótulo usado no comando \autoref{}
\renewcommand{\lstlistingname}{Código}
% Altera o rótulo a ser usando no elemento pré-textual "Lista de código"
\renewcommand{\lstlistlistingname}{Lista de códigos}

% Configura a ``Lista de Códigos'' conforme as regras da ABNT (para abnTeX2)
\begingroup\makeatletter
\let\newcounter\@gobble\let\setcounter\@gobbletwo
  \globaldefs\@ne \let\c@loldepth\@ne
  \newlistof{listings}{lol}{\lstlistlistingname}
  \newlistentry{lstlisting}{lol}{0}
\endgroup

\renewcommand{\cftlstlistingaftersnum}{\hfill--\hfill}

\let\oldlstlistoflistings\lstlistoflistings
\renewcommand{\lstlistoflistings}{%
   \begingroup%
   \let\oldnumberline\numberline%
   \renewcommand{\numberline}{\lstlistingname\space\oldnumberline}%
   \oldlstlistoflistings%
   \endgroup}

\definecolor{codegreen}{rgb}{0,0.6,0}
\definecolor{codegray}{rgb}{0.5,0.5,0.5}
\definecolor{codepurple}{rgb}{0.58,0,0.82}
\definecolor{backcolour}{rgb}{0.95,0.95,0.92}
\definecolor{delim}{RGB}{20,105,176}
\definecolor{numb}{RGB}{106, 109, 32}
\definecolor{string}{rgb}{0.64,0.08,0.08}

\lstdefinestyle{estiloCodigos}{
    backgroundcolor=\color{backcolour},   
    commentstyle=\color{codegreen},
    keywordstyle=\color{magenta},
    numberstyle=\tiny\color{codegray},
    stringstyle=\color{codepurple},
    basicstyle=\ttfamily\footnotesize,
    breakatwhitespace=false,         
    breaklines=true,                 
    captionpos=b,                    
    keepspaces=true,                 
    numbers=left,                    
    numbersep=5pt,                  
    showspaces=false,                
    showstringspaces=false,
    showtabs=false,                  
    tabsize=2
}

\lstset{escapechar=@,style=estiloCodigos}


\titulo{Arquiteturas para processamento de dados: Comparativo entre arquiteturas \textit{Lambda} e arquiteturas \textit{Kappa}}
\autor{Kauan Amarante}
\local{Araquari -- SC}
\data{2022}
\orientador{Prof. Dr. Eduardo da Silva}
% Informações de dados para CAPA e FOLHA DE ROSTO
\instituicao{%
  Instituto Federal Catarinense -- IFC
  \par
  Campus Araquari
  \par
  Bacharelado em Sistemas de Informação}
\tipotrabalho{Monografia (Graduação)}
% O preambulo deve conter o tipo do trabalho, o objetivo, 
% o nome da instituição e a área de concentração 
\preambulo{Trabalho de conclusão de curso apresentado como requisito parcial para a obtenção do grau de bacharel em Sistemas de Informação do Instituto Federal Catarinense.}


% Configurações de aparência do PDF final
% informações do PDF
\makeatletter
\hypersetup{
   	%pagebackref=true,
		pdftitle={\@title}, 
		pdfauthor={\@author},
   	pdfsubject={\imprimirpreambulo},
    pdfcreator={LaTeX with abnTeX2},
		pdfkeywords={abnt}{latex}{abntex}{abntex2}{trabalho acadêmico}, 
		colorlinks=true,       		% false: boxed links; true: colored links
   	linkcolor=blue,          	% color of internal links
   	citecolor=blue,        		% color of links to bibliography
   	filecolor=magenta,     		% color of file links
		urlcolor=blue,
		bookmarksdepth=4
}
\makeatother

% Espaçamentos entre linhas e parágrafos 

\setlength{\parindent}{1.3cm}			% O tamanho do parágrafo é dado por:
\setlength{\parskip}{0.2cm}  			% Controle do espaçamento entre um parágrafo e outro:


% Início do documento
\begin{document}
%\selectlanguage{english}
\selectlanguage{brazil} 				% Seleciona o idioma do documento (conforme pacotes do babel)
\frenchspacing 							    % Retira espaço extra obsoleto entre as frases.

% ELEMENTOS PRÉ-TEXTUAIS
\imprimircapa							% Capa
\imprimirfolhaderosto*		% Folha de rosto
								          % (o * indica que haverá a ficha bibliográfica)

% Inserir a ficha bibliografica

% Isto é um exemplo de Ficha Catalográfica, ou ``Dados internacionais de 
% catalogação-na-publicação''. Utilizar este modelo como referência. 
% Porém, provavelmente a biblioteca fornecerá um PDF com a ficha catalográfica 
% definitiva após a defesa do trabalho. Quando estiver com o documento, salve-o como PDF 
% no diretório do seu projeto e substitua todo o conteúdo de implementação deste arquivo 
% pelo comando abaixo:
%
% \begin{fichacatalografica}
%     \includepdf{fig_ficha_catalografica.pdf}
% \end{fichacatalografica}

\begin{fichacatalografica}
	\sffamily
	\vspace*{\fill}					% Posição vertical
	\begin{center}					% Minipage Centralizado
	\fbox{\begin{minipage}[c][8cm]{13.5cm}		% Largura
	\small
	\imprimirautor
	
	
	\hspace{0.5cm} \imprimirtitulo   / \imprimirautor. --
	\imprimirlocal, \imprimirdata-
	
	\hspace{0.5cm} \pageref{LastPage} p. : il. (algumas color.) ; 30 cm.\\
	
	\hspace{0.5cm} \imprimirorientadorRotulo~\imprimirorientador\\
	
	\hspace{0.5cm}
	\parbox[t]{\textwidth}{\imprimirtipotrabalho~--~\imprimirinstituicao,
	\imprimirdata.}\\
	
	\hspace{0.5cm}
		1. Arquitetura.
		2. Processamento de dados.
		3. \textit{Lambda}.
		4. \textit{Kappa}.
		I. Eduardo da Silva.
		II. Instituto Federal Catarinense.
		III. Câmpus Araquari.
		IV. Arquiteturas para processamento de dados: Comparativo entre arquiteturas \textit{Lambda} e arquiteturas \textit{Kappa} 			
	\end{minipage}}
	\end{center}
\end{fichacatalografica}
% ---



% Inserir folha de aprovação

% Exemplo de Folha de aprovação, elemento obrigatório da NBR 14724/2011 (seção 4.2.1.3). 
% Utilizar este modelo até a aprovação do trabalho. Após isso, substitua todo o conteúdo 
% deste arquivo por uma imagem da página assinada pela banca com o comando abaixo:
%
% \includepdf{folhadeaprovacao_final.pdf}
%
\begin{folhadeaprovacao}

  \begin{center}
    {\ABNTEXchapterfont\large\imprimirautor}

    \vspace*{\fill}\vspace*{\fill}
    \begin{center}
      \ABNTEXchapterfont\bfseries\Large\imprimirtitulo
    \end{center}
    \vspace*{\fill}
    
    \hspace{.45\textwidth}
    \begin{minipage}{.5\textwidth}
        \imprimirpreambulo
    \end{minipage}%
    \vspace*{\fill}
   \end{center}
        
    Trabalho aprovado. \imprimirlocal, 24 de novembro de 2016:

   \assinatura{\textbf{\imprimirorientador} \\ Orientador} 
   \assinatura{\textbf{Professor} \\ Convidado 1}
   \assinatura{\textbf{Professor} \\ Convidado 2}
   %\assinatura{\textbf{Professor} \\ Convidado 3}
   %\assinatura{\textbf{Professor} \\ Convidado 4}
      
   \begin{center}
    \vspace*{0.5cm}
    {\large\imprimirlocal}
    \par
    {\large\imprimirdata}
    \vspace*{1cm}
  \end{center}
  
\end{folhadeaprovacao}
% ---

% Dedicatória
\begin{dedicatoria}
   \vspace*{\fill}
   \centering
   \noindent
   \textit{ A todos aqueles que de alguma forma estiveram e estão próximos de mim, fazendo esta vida valer cada vez mais a pena.} \vspace*{\fill}
\end{dedicatoria}


% Agradecimentos
\begin{agradecimentos}

\end{agradecimentos}

% Epígrafe
\begin{epigrafe}
    \vspace*{\fill}
	\begin{flushright}
		\textit{``Entre o nascer e o morrer existe algo, \\
        um momento efêmero, único, de uma \\
        fascínio estúpido, que se chama vida''\\ (Lourenço Mutarelli)}
	\end{flushright}
\end{epigrafe}
% RESUMOS
% resumo em português
\setlength{\absparsep}{18pt} % ajusta o espaçamento dos parágrafos do resumo
\begin{resumo}
 Segundo a \citeonline[3.1-3.2]{NBR6028:2003}, o resumo deve ressaltar o
 objetivo, o método, os resultados e as conclusões do documento. A ordem e a extensão
 destes itens dependem do tipo de resumo (informativo ou indicativo) e do
 tratamento que cada item recebe no documento original. O resumo deve ser
 precedido da referência do documento, com exceção do resumo inserido no
 próprio documento. (\ldots) As palavras-chave devem figurar logo abaixo do
 resumo, antecedidas da expressão Palavras-chave:, separadas entre si por
 ponto e finalizadas também por ponto.

 \textbf{Palavras-chave}: latex. abntex. editoração de texto.
\end{resumo}

% resumo em inglês
\begin{resumo}[Abstract]
 \begin{otherlanguage*}{english}
   This is the english abstract.

   \vspace{\onelineskip}
 
   \noindent 
   \textbf{Keywords}: latex. abntex. text editoration.
 \end{otherlanguage*}
\end{resumo}


% ---
% inserir lista de ilustrações
% ---
\pdfbookmark[0]{\listfigurename}{lof}
\listoffigures*
\cleardoublepage
% ---

% ---
% inserir lista de tabelas
% ---
\pdfbookmark[0]{\listtablename}{lot}
\listoftables*
\cleardoublepage
% ---
% inserir lista de codigos
% ---
\pdfbookmark[0]{\lstlistlistingname}{lol}
\begin{KeepFromToc}
\lstlistoflistings
\end{KeepFromToc}
\cleardoublepage
% ---

% ---
% inserir lista de abreviaturas e siglas
% ---
\begin{siglas}
  \item[ABNT] Associação Brasileira de Normas Técnicas
  \item[abnTeX] ABsurdas Normas para TeX
\end{siglas}
% ---

% ---
% inserir lista de símbolos
% ---
\begin{simbolos}
  \item[$ \Gamma $] Letra grega Gama
  \item[$ \Lambda $] Lambda
  \item[$ \zeta $] Letra grega minúscula zeta
  \item[$ \in $] Pertence
\end{simbolos}
% ---

% ---
% inserir o sumario
% ---
\pdfbookmark[0]{\contentsname}{toc}
\tableofcontents*
\cleardoublepage
% ---

% ----------------------------------------------------------
% ELEMENTOS TEXTUAIS
% ----------------------------------------------------------
\textual

% ---
% Inclusão de capítulo de Introdução ao trabalho
% ---
\chapter[Introdução]{Introdução}

\chapter{Referencial Teórico}

Para o desenvolvimento deste projeto, fez-se necessária a realização de uma revisão de literatura, 
de modo a obter uma base sólida de conhecimento para a condução das atividades pertinentes a este trabalho, 
e seguem descritas na sequência.

\section{\textit{Big Data}}
 \cite{sagiroglu2013big}

\section{Arquiteturas para processamento de dados}
\subsection{Arquitetura \textit{Lambda}}
\subsection{Arquitetura \textit{Kappa}}

\section{Apache Kafka}

\section{\textit{Data Lake}}
\chapter{Metodologia}

Com o objetivo de garantir uma infraestrutura robusta e escalável para a aplicação escolhida, ficou decidido pela utilização de um \textit{cluster} Kubernetes, visto sua capacidade de resiliência e alta disponibilidade.


\section{Área de Estudo}



\section{Método}

\textbf{Alterar inicio} Com o intuito de realizar uma prova de conceito de baixo custo, optou-se pela utilização de uma distribuição Kubernetes local, chamada de Minikube, 




\chapter{Resultados}

\chapter{}

% ----------------------------------------------------------
% Finaliza a parte no bookmark do PDF
% para que se inicie o bookmark na raiz
% e adiciona espaço de parte no Sumário
% ----------------------------------------------------------
\phantompart

\chapter{Considerações Finais}
A PoC teve grande valia, uma vez que mostrou a 
ineficácia do Hudi ao tratar dados particionados na origem, além de somente funcionar no master node do EMR. Talvez seja por ser algo que ainda esteja em incubação pela Apache Software Foundation e em um futuro possam resolver estes problemas, por este motivo foi decidido dar a PoC como encerrada.

% ----------------------------------------------------------
% ELEMENTOS PÓS-TEXTUAIS
% ----------------------------------------------------------
\postextual
% ----------------------------------------------------------

% ----------------------------------------------------------
% Referências bibliográficas
% ----------------------------------------------------------
\bibliography{referencias}

% ----------------------------------------------------------
% Apêndices
% ----------------------------------------------------------

% ---
% Inicia os apêndices
% ---
 \let\tccaddcontentsline\addcontentsline
\renewcommand\addcontentsline[3]{
  \ifthenelse{\equal{#1}{lof}}{} {
    \ifthenelse{\equal{#1}{lot}}{}{\tccaddcontentsline{#1}{#2}{#3}}
  }
}
\begin{apendicesenv}

% Imprime uma página indicando o início dos apêndices
\partapendices

\chapter{Código para realizar a inserção de dados} \label{apendice}

\begin{lstlisting}
package com.contaazul.jarvis.hudi.insert

import org.apache.hudi.DataSourceWriteOptions
import org.apache.hudi.config.HoodieWriteConfig
import org.apache.hudi.hive.MultiPartKeysValueExtractor
import org.apache.spark.sql.functions.{concat, lit}
import org.apache.spark.sql.types.DateType
import org.apache.spark.sql.{SaveMode, SparkSession}
import org.apache.spark.{SparkConf, SparkContext}

object HudiInsert {

  def processInsert(params: Map[String, String]) {

    val spark = SparkSession.builder().appName("insert " + params("table")).getOrCreate()

    import spark.implicits._

    // Read data from S3 and create a DataFrame with Partition and Record Key
    var insertDF = spark.read.json(
      "s3://cdc-tests/cdc/" + params("upsertOption") + "/" +
        params("instance") + "/" + params("database") + "/" +
        params("schema") + "/" + params("table") + "/" +
        params("year") + "/" + params("month") + "/" +
        params("day") + "/" + params("hour"))

    insertDF = insertDF.withColumn("committed_at", insertDF("committed_at").cast(DateType))

    val hudiTablePartitionKey = "partition_key"
    insertDF = insertDF.withColumn(hudiTablePartitionKey, concat(lit("year="), $"year", lit("/month="), $"month", lit("/day="), $"day", lit("/hour="), $"hour"))

    //Specify common DataSourceWriteOptions in the single hudiOptions variable
    val hudiOptions = Map[String, String](
      HoodieWriteConfig.TABLE_NAME -> params("table"),
      DataSourceWriteOptions.STORAGE_TYPE_OPT_KEY -> "COPY_ON_WRITE",
      DataSourceWriteOptions.RECORDKEY_FIELD_OPT_KEY -> "id",
      DataSourceWriteOptions.PARTITIONPATH_FIELD_OPT_KEY -> hudiTablePartitionKey,
      DataSourceWriteOptions.PRECOMBINE_FIELD_OPT_KEY -> "committed_at",
      DataSourceWriteOptions.HIVE_SYNC_ENABLED_OPT_KEY -> "true",
      DataSourceWriteOptions.HIVE_DATABASE_OPT_KEY -> params("database"),
      DataSourceWriteOptions.HIVE_TABLE_OPT_KEY -> params("table"),
      DataSourceWriteOptions.HIVE_PARTITION_FIELDS_OPT_KEY -> "year,month,day,hour",
      DataSourceWriteOptions.HIVE_PARTITION_EXTRACTOR_CLASS_OPT_KEY -> classOf[MultiPartKeysValueExtractor].getName
    )

    insertDF.write.format("org.apache.hudi")
      .option(DataSourceWriteOptions.OPERATION_OPT_KEY, DataSourceWriteOptions.INSERT_OPERATION_OPT_VAL)
      .options(hudiOptions)
      .mode(SaveMode.Overwrite)
      .save("s3://cdc-tests/hudi/" + params("table"))
  }

  def main(args: Array[String]) {

    val conf = new SparkConf().setAppName("Update Users")
    conf.set("spark.serializer", "org.apache.spark.serializer.KryoSerializer")
    conf.set("spark.sql.hive.convertMetastoreParquet", "false")
    conf.set("spark.executor.memory", "7G")
    conf.set("spark.dynamicAllocation.executorIdleTimeout", "3600")
    conf.set("spark.executor.cores", "1")
    conf.set("spark.dynamicAllocation.initialExecutors", "16")
    conf.set("spark.sql.parquet.outputTimestampType", "TIMESTAMP_MILLIS")

    val sc = new SparkContext(conf)

    val instance = args(0)
    val database = args(1)
    val schema = args(2)
    val table = args(3)
    val year = args(4)
    val month = args(5)
    val day = args(6)
    val hour = args(7)

    val params = Map[String, String](
      "instance" -> instance,
      "database" -> database,
      "schema" -> schema,
      "table" -> table,
      "year" -> year,
      "month" -> month,
      "day" -> day,
      "hour" -> hour
    )

    processInsert(params)

    sc.stop()
  }
}

\end{lstlisting}

\chapter{Código para realizar a atualização e exclusão de dados} \label{apendice}

\begin{lstlisting}
package com.contaazul.jarvis.hudi.insert

import org.apache.hudi.DataSourceWriteOptions
import org.apache.hudi.config.HoodieWriteConfig
import org.apache.hudi.hive.MultiPartKeysValueExtractor
import org.apache.spark.sql.functions.{concat, lit}
import org.apache.spark.sql.types.DateType
import org.apache.spark.sql.{SaveMode, SparkSession}
import org.apache.spark.{SparkConf, SparkContext}

object HudiInsert {

  def processInsert(params: Map[String, String]) {

    val spark = SparkSession.builder().appName("insert " + params("table")).getOrCreate()

    import spark.implicits._

    // Read data from S3 and create a DataFrame with Partition and Record Key
    var insertDF = spark.read.json(
      "s3://cdc-tests/cdc/" + params("upsertOption") + "/" +
        params("instance") + "/" + params("database") + "/" +
        params("schema") + "/" + params("table") + "/" +
        params("year") + "/" + params("month") + "/" +
        params("day") + "/" + params("hour"))

    insertDF = insertDF.withColumn("committed_at", insertDF("committed_at").cast(DateType))

    val hudiTablePartitionKey = "partition_key"
    insertDF = insertDF.withColumn(hudiTablePartitionKey, concat(lit("year="), $"year", lit("/month="), $"month", lit("/day="), $"day", lit("/hour="), $"hour"))

    //Specify common DataSourceWriteOptions in the single hudiOptions variable
    val hudiOptions = Map[String, String](
      HoodieWriteConfig.TABLE_NAME -> params("table"),
      DataSourceWriteOptions.STORAGE_TYPE_OPT_KEY -> "COPY_ON_WRITE",
      DataSourceWriteOptions.RECORDKEY_FIELD_OPT_KEY -> "id",
      DataSourceWriteOptions.PARTITIONPATH_FIELD_OPT_KEY -> hudiTablePartitionKey,
      DataSourceWriteOptions.PRECOMBINE_FIELD_OPT_KEY -> "committed_at",
      DataSourceWriteOptions.HIVE_SYNC_ENABLED_OPT_KEY -> "true",
      DataSourceWriteOptions.HIVE_DATABASE_OPT_KEY -> params("database"),
      DataSourceWriteOptions.HIVE_TABLE_OPT_KEY -> params("table"),
      DataSourceWriteOptions.HIVE_PARTITION_FIELDS_OPT_KEY -> "year,month,day,hour",
      DataSourceWriteOptions.HIVE_PARTITION_EXTRACTOR_CLASS_OPT_KEY -> classOf[MultiPartKeysValueExtractor].getName
    )

    insertDF.write.format("org.apache.hudi")
      .option(DataSourceWriteOptions.OPERATION_OPT_KEY, DataSourceWriteOptions.INSERT_OPERATION_OPT_VAL)
      .options(hudiOptions)
      .mode(SaveMode.Overwrite)
      .save("s3://cdc-tests/hudi/" + params("table"))
  }

  def main(args: Array[String]) {

    val conf = new SparkConf().setAppName("Update Users")
    conf.set("spark.serializer", "org.apache.spark.serializer.KryoSerializer")
    conf.set("spark.sql.hive.convertMetastoreParquet", "false")
    conf.set("spark.executor.memory", "7G")
    conf.set("spark.dynamicAllocation.executorIdleTimeout", "3600")
    conf.set("spark.executor.cores", "1")
    conf.set("spark.dynamicAllocation.initialExecutors", "16")
    conf.set("spark.sql.parquet.outputTimestampType", "TIMESTAMP_MILLIS")

    val sc = new SparkContext(conf)

    val instance = args(0)
    val database = args(1)
    val schema = args(2)
    val table = args(3)
    val year = args(4)
    val month = args(5)
    val day = args(6)
    val hour = args(7)

    val params = Map[String, String](
      "instance" -> instance,
      "database" -> database,
      "schema" -> schema,
      "table" -> table,
      "year" -> year,
      "month" -> month,
      "day" -> day,
      "hour" -> hour
    )

    processInsert(params)

    sc.stop()
  }
}

\end{lstlisting}

\end{apendicesenv}
\end{document}
